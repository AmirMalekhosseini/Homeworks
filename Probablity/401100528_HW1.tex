\documentclass[12pt]{article}
\usepackage{authblk} % Add the authblk package for author affiliations

\title{Probablity Homework 1}
\author{Amir Malekhosseini}
\affil{Student id: 401100528}
\affil{Department of Mathematical Science, Sharif University Of Technology}

\begin{document}
\maketitle

\section*{Problem 1}
We want to know the Probablit of sum 5.
\newline The first we should do is to create the sample space for the problem.
\newline We stop throwing only if the sum of two dices is 5 or 7
it means that if we draw the desicion tree for this problem
it will stop in some position(when sum 5 or 7 happens) or it will continue infinitly.
\newline So the smaple space is:
\newline $\Omega = \{(1,4),(4,1),(2,3),(3,2),(3,4),(4,3)(1,6),(6,1),(5,2),(2,5)\} $
\newline We know that the sample space will happen because if we throw dices infinite times,
the Probablity of not showing sum 5 or 7 is that every time the sum happens something else and then
the Probablity is $ \liminf\frac{1}{n} $ and is zero.
\newline So the Probablity of the sample space we wrote is 1. So we can limit or whole problem's sample space to $\Omega$
(Because the 3 conditions for a Probablity sample space that is in book is correct for this $\Omega$ so we can accept it).
\newline Now that we have chosen our limited sample space, for answering to problem we can easily count Probablity of showing sum 5, that is 4.
\newline And our $\Omega$ had 10 members so the final Probablity is $\frac{4}{10}$.

\section*{Problem 2}
I claim that the Probablity is $\frac{1}{4}$.The prove is:
\newline First thing we should notice is that the three pieces can form a triangle if and only if the triangle inequalities hold:
\newline
\begin{verbatim}
    "The sum of the lengths of any two pieces must be
    greater than the length of the third piece."
\end{verbatim}

We can write a desicion three and calculate the Probablity at each point.\
\newline But the efficeint way is to use a geometary theorem that says:
\begin{verbatim}
    " If we have a 	Equilateral triangle ,
    the distance of each point in the triangle to the triangle's sides 
    is equal to height of the triangle "
\end{verbatim}

Because when we cut our first position(suppose i n) on the piece of wood, the
first side of triangle we want to make will be showed and the sum of two others
sides is $ 1 - n $. \newline So we can simulate the probelm to choosing a point
in a Equilateral triangle with height 1 (wood height) and use the geometary
theorem said earlier. \newline So we create a Equilateral triangle with height
1 and mark its sides's middle points and connect them. \newline Base on another
geometary theorem 4 equal triangles will be created. \newline If we pay
attention to these triangles we see that each point we choose in middle
triangle, it matches the condition we wanted to create a triangle with piece of
wood(other 3 triangles don't satisfie the condition) and if we suppose that the
point has distances a,b,c to sides, as the sum of a,b,c is 1(base on earlier
theorem) and it satisfies the condition we wanted so we can choose the a,b,c as
our disered points on the peice of wood and create a triangle with it.\newline
So our sample space is $\Omega = \{$4 triangles with equal area$\}$,and we has
the answer only if we choose a point in the middle triangle and because all
triangles has equal area that is 1 out of 4, so the final Probablity is
$\frac{1}{4}$ as claimed.

\section*{Problem 3}
For this problem we use the theorem that says $ \Pr[A] = 1 - \Pr[A'] $.
And for calculate Probablity that none of S divides 3,
we calculate Probablity that each S does divide 3,
and then use the theorem to calculate the Probablity that each S doesn't divide 3,
and because we want all these situations happen together, we multipy all Probablities .
\newline We set Pr[$S_{n}$] = \{$S_{n} divides 3$ \}.
\newline So base on the theorem Pr[$S'_{n}$] = 1 - Pr[$S_{n}$].
\newline\newline $\Rightarrow 1 - \Pr[S_{1}] = 1 - \frac{\left\lfloor \frac{n}{3} \right\rfloor}{n}$
\newline $\Rightarrow 1 - \Pr[S_{2}] = 1 - \frac{\left\lfloor \frac{n - 1}{3} \right\rfloor}{n - 1}$
\newline $\Rightarrow \cdots$
\newline $\Rightarrow \cdots$
\newline $\Rightarrow 1 - \Pr[S_{n}] = 1 - 0 = 1$
\newline \newline If we continue the calculation we can find each $\Pr[S'_{n}]$
and for final answer we multipy them together.

\section*{Problem 4}
The last passenger wouldn't able to sit on his own sit if and only if it is not empty.
It means that someone else has already sitten there.
So we try to simulate the conditions by using a desicion tree.
\newline For first passenger there is two conditions:
\newline\newline 1- sits on his own sit with Probablity $ \frac{1}{200} $.
\newline 2- sits on someone else's sit with Probablity $\frac{199}{200}$.
\newline\newline If the first condition happens, all passengers will sit on their own sit as well as the last one.
\newline But if the second condition happens, and suppose the first passenger sits on jth sit
then all 2 to j-1th passengers will sit on their own sit and the jth passengers again has two situations:
\newline\newline 1- Sit on first sit with Probablity $\frac{1}{200-(j-1)}$.
\newline 2 Sit on another sit.
\newline\newline Just like before if the first situation happens the last passenger will sit on his own sit
and if not the desicion tree continues.
\newline If we continue this method and sum Probablities(because there is "or" between situations)
then we get to $\frac{1}{2}$ for the final answer.
\newline It means the last passenger has $\frac{1}{2}$ to sit on his own sit.

\section*{Problem 5}
Like we have in context of the question, we set $A_{m}$
as area with distance between $mr_{0}$ and $(m-1)r_{0}$.
\newline Therefore we have:
\newline $A_{m} = \{$points with $(m-1)r_{0}$ < d < $mr_{0}$$\}$
        \newline Now we try to calculate each $A_{m}$:
        \newline \newline $\Rightarrow A_{1} = \{$points with 0 < d < $r_{0}$$\}$
    \newline $\Rightarrow A_{2} = \{$points with $r_{0}$ < d < $2r_{0}$$\}$
    \newline $\Rightarrow\cdots$
    \newline \newline Because we had a relationship between Probablities in context of question
    therefore we can set $\Pr[A_{1}] = \alpha$ and write other Probablities base on $\alpha$.
    \newline so we have:
    \newline \newline $\Rightarrow A_{1}=\alpha, A_{2}= \frac{\alpha}{2}, \ldots$
    \newline $\Rightarrow A_{1} + A_{2} + \ldots = 1 $
    \newline $\Rightarrow \alpha = \frac{1}{2}$
    \newline \newline Now we are going to calculate the Probablity of square:
    \newline The square will be inside a circle with radius $2r_{0}$ (call it $S_{2}$).
    \newline And also there is a circle with radius $r_{0}$ in the square(call it $S_{1}$).
    \newline If we pay attention we see that these $S_{1} , S_{2}$ are $A_{1} , A_{2}$ that we calculated earlier.
    \newline So base on these explainations if a shot wants to be inside the square
    it is inside the $S_{1}$ 'or' inside the area between $S_{1}$ and the square.
    \newline As we calculated $S_{1}$ before (it is eqaul to $\alpha$)
    So for calculating the Probablity of shooting in square, all we need to do is
    calculating Probablity of area between the $S_{1}$ and the square itself(by subtracting areas) and then sum it with $\alpha$.
    \newline After calculating every area($S_{1}, S_{2}$, the square) the subtracted area is: $\frac{4-\pi}{3\pi}$
    \newline So the Probablity that shot be inside area between $S_{1}$ and the square is: $ \frac{1}{4} * \frac{4-\pi}{3\pi}$
    \newline \newline Therefore the fianl Probablity is: $ \frac{1}{2} + \frac{1}{4} * \frac{4-\pi}{3\pi} $

\end{document}